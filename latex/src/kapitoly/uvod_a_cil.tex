\kapitola{Úvod a cíl}
\sekce{Úvod}
Společnosti poskytující služby v oblasti Location Intelligence využívají své platformy pro komplexní geoprostorové analýzy, což je zásadní pro efektivní rozhodování v různých sektorech podnikání. Jako součást těchto platforem některé společnosti, včetně CleverMaps, nabízí služby zvané jako Data Marketplace, které umožňují uživatelům získávat a integrovat různorodé datové sady. Takový marketplace poskytuje širokou škálu dat. Příkladem takových dat mohou být demografické informace, které klientům umožňují získat přesné informace například o jejich cílové skupině. \cite{clevermaps_location_2024}

Vzhledem k tomu, že při práci (nejen) s geodaty se často zapracovávají informace získané z různých zdrojů, nastává to, že se totožné údaje ve shromážděných datových sadách mohou objevovat vícekrát. Tento jev se označuje jako duplicity. Tyto duplicity mohou nastat z mnoha důvodů a mohou mít různé podoby. Příkladem může být situace, kdy jsou v různých databázích stejná místa nebo objekty zaznamenány pod různými názvy. Například, jedna databáze může obsahovat bod zájmu (point of interest) jako "Velký městský park", zatímco druhá jej může uvádět jako "Park v centru města". Obě tato označení se mohou vztahovat ke stejnému místu, ale různé názvy komplikují jejich rozpoznání jako duplicit. Další běžnou příčinou je chybné zadání dat, kde se například dva různé záznamy pro stejnou ulici mohou lišit pouze malými typografickými chybami nebo odlišnými způsoby zkrácení názvů. \cite{nauman_introduction_2022}

Pro zajištění integrity a přesnosti geodat je detekce a řešení duplicit nezbytná. K identifikaci duplicit mohou být využity různé metody, od jednoduchých porovnání řetězců až po sofistikovanější techniky, jako jsou algoritmy strojového učení a neuronové sítě. Tyto pokročilé metody umožňují rozpoznávat a porovnávat podobnosti na základě kontextu a pravděpodobnosti, což vede k efektivnějšímu a přesnějšímu detekování duplicit. \cite{christen_data_2012}

V oblasti detekce duplicit existuje řada služeb, které nabízejí pokročilé řešení tohoto problému. Tyto služby jsou často založené na cloud technologiích a strojovém učení. Jsou schopné efektivně identifikovat a eliminovat duplicity v rozsáhlých datových sadách. Mezi poskytovatele těchto služeb patří například služby jako Data Ladder, Tilores nebo Melissa, které nabízejí různé nástroje pro automatickou detekci duplicitních záznamů.

Ačkoliv tyto služby přinášejí významné výhody v podobě úspory času a zlepšení kvality dat, mohou být spojeny s vysokými náklady, zejména v případech, kdy je zapotřebí zpracovat velké objemy dat. Finanční zátěž z těchto služeb může být pro některé společnosti značná, což může motivovat k implementaci vlastního řešení.\cite{christen_data_2012}

\sekce{Cíl}
Cílem této práce je prozkoumat a otestovat různé metody detekce duplicit na geoprostorových datech, včetně metod založených na strojovém učení. Na základě analýzy výsledků testů doporučit nejvhodnější metody pro konkrétní typy sad geoprostorových dat, přičemž ověření těchto metod proběhne na datových sadách poskytnutých společností CleverMaps.

Výsledná doporučení by měla společnosti CleverMaps pomoci v rámci zvyšování automatizace a zkvalitnění procesů kontrol kvality dat.
