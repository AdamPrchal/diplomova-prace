\kapitola{Literární rešerše}


\sekce {Problematika detekce duplicit v datech}
\todo{Základní pojmy, metriky, typy algoritmů, problémy s kvalitou dat}

\sekce {Přístupy k rozpoznání duplicitních objektů}
\todo{Algoritmy a techniky}

\sekce {Geoprostorová data}
\todo {Typy geodat(vector, body, linie a rastry), formáty uložení, souřadnicové systémy}
Jde o data, která představují místa, oblasti nebo třeba předměty ze skutečného světa, a udávají jejich přesnou lokalitu pomocí souřadnic na Zemi. Můžeme mít například geodata představující státní hranice evropských zemí, geodata představující všechny lampy veřejného osvětlení v Brně.
\todo{Obrázek s GIS mapama}

\podsekce{Souřadnicové systémy}

Geodata používají souřadnice pro popsání pozice konkrétního objektu. Těchto souřadnic však existuje hned několik a při práci s geodaty je potřeba vědět, které to jsou. Existují různé typy souřadnicových systémů, které se používají podle konkrétních požadavků. Mezi nejpoužívanější v kontextu České republiky patří:

\podsekce{Globální a celosvětové použití}

\podsekce{Geografický souřadnicový systém (GCS) - WGS 84}

Systém využívající úhlových souřadnic, konkrétně zeměpisné šířky a délky. Nadmořská výška se udává v metrech nad povrchem referenčního elipsoidu.
Je vhodný pro geodata představující entity na celosvětové úrovni. Díky tomu je využíván například i v GPS technologii, kde je zapotřebí pracovat v rámci celé Země.
Není však příliš vhodná pro lokální úroveň (města, ulice), pokud je vyžadována vysoká přesnost. Je tomu tak protože při výpočtu vzdáleností nebo ploch z úhlových souřadnic může docházet k nepřesnostem.

\podsekce{Regionální a národní použití}

\podsekce{Projekční souřadnicový systém (PCS) - S-JTSK (Systém Jednotné Trigonometrické Sítě Katastrální)}

Tento konkrétní systém je navržen pro Českou republiku a umožňuje tak na jejím území mapovat s vysokou přesností. Je proto používán jako jeden ze závazných souřadnicových systémů pro orgány České republiky.
Souřadnice jsou vyjádřeny pomocí kartézského souřadnicového systému (X - sever, Y - východ) a jednotkou jsou metry. Pro nadmořskou výšku používá metry nad střední hladinou Jaderského moře.

\podsekce{Evropský souřadnicový systém - ETRS89 (European Terrestrial Reference System 1989)}

Souřadnicový systém ETRS89 slouží jako standard pro evropské projekty a díky tomu i usnadňuje spolupráci vícero zemí na projektech s mezihraničním rozsahem.
Stejně jako S-JTSK používá kartézský souřadnicový systém v jednotkách metru. Výška v tomto systému je počítáná v metrech od povrchu referenčního elipsoidu GRS80.

Toto nejsou zdaleka všechny souřadnicové systémy. Každý z nich slouží pro jiné potřeby a je důležité si uvědomit, jak se liší, a v jakém kontextu se s nimi lze setkat.

\todo{Tvary geodat (points, lines, and polygons)}

\podsekce{K čemu slouží}

Geodata sama o sobě ale popisují vždy specifický objekt. Aby se dali z těchto dat získat nějaké znalosti, je potřeba je dále zpracovávat. K tomu slouží například geografické informační systémy (GIS). Tyto systémy dokáží poskytnutá data analyzovat a vizualizovat a dále s nimi pracovat dle potřeby. Mohou tak umožnit například přehledně vidět, v jakém úseku silnice je v daný čas nejvíce aut, a zároveň vidět, ze kterých navazujících silnic tam tyto auto přijíždí. Díky takové vizualizaci a analýze lze následně vycházet například při plánování dopravní infrastruktury.

Každý kdo otevře služby jako Google Maps, Seznam Mapy nebo Apple Maps se dívá na několika vrstvou vizualizaci geodat. Na první pohled se jedná o běžnou mapu, avšak vizualizace v těchto službách je běžně tvořena několika samostatnými vrstvami:

\begin{itemize}
    \item vrstvy komunikace,
    \item vrstvy staveb,
    \item vrstvy řek,
    \item vrstvy zeleně a dalších.
\end{itemize}

\todo{Obrázek map z aplikací}
Každá taková vrstva je sada geodat, která popisuje daný typ předmětu/lokace s přesnými souřadnicemi dalšími pomocnými atributy/vlastnostmi.

\sekce {Existující nástroje}
\todo{Popište dostupné software, jejich výhody a nevýhody}
